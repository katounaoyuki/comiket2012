\part{応用編:関数型の機能を学ぶ}
\begin{itembox}[l]{ここで学ぶこと}
\begin{itemize}
\item 第一級オブジェクトとしての関数
\item パターンマッチ(match式) 
\item ケースクラス
\item トレイト
\end{itemize}
\end{itembox}

応用編ではまず、関数型言語の「関数」とはそもそもどういう概念かを説明します。関数は、一つ以上の引数を取り、一つの結果(戻り値)を生成する役割を持つオブジェクトです。基礎編で登場したメソッド\footnote{defキーワードで宣言するメソッドのことです。} はクラスに従属しますが、関数は単独で利用できます。例えば、
\begin{lstlisting}[language=scala, frame=none]
(x:Int) => x * x
\end{lstlisting}
は、「関数リテラル」と呼ばれる記述方法で、この例はIntの引数を取り、それを二乗した結果(戻り値)を生成する関数を示します。また、この関数リテラルで記述される名前のない関数を「無名関数」といいます。 

\subsection{関数を変数に代入したり引数で渡せる}
関数型言語での関数は、他の値と同様、変数に代入したり、別の関数の引数として渡したり、関数の戻り値として返したりと、基本的な操作を制限なく行える一人前のオブジェクトです。この性質を持つオブジェクトを「第一級オブジェクト」といいます。 例えば、\lstRef{src:variable_function}は関数を変数に代入する例です。 squareに代入された関数は、Int型の引数と、Int型の戻り値を持つ型として、定義されたことがわかります。定義した関数は、\lstRef{src:function_call}のようにして呼び出せます。定義した引数や戻り値に間違った型を指定するとコンパイルエラーとなります。「型安全」(タイプセーフ)なので、プログラマの負担が軽減されます。関数はまた、\lstRef{src:higher_order_function}のように通常の値と同様、関数の引数や戻り値に利用できます。このように関数を利用する関数のことを「高階関数」と呼びます。高階関数を利用することで、命令型言語では表現できないシンプルなコードが記述可能です。

\begin{lstlisting}[language=scala, label=src:variable_function, caption=関数を変数に代入する例]
// val 変数名 = 引数のリスト => コードブロック
val square = (x:Int) => x * x
// コードブロックが複数行の場合({ }が必要)
val squareDebug = (x:Int) => {
  val result = x * x println(result) result
} 
\end{lstlisting}

\begin{lstlisting}[language=scala, label=src:function_call, caption=値としての関数を呼び出す]
val result1 = square(2)		// result1 = 4
val result2 = squareDebug(2)    // result2 = 4 標準出力に4を表示
val result3 = square("aaa")	// 第一引数の型がコンパイルエラー
val result4:String = square(2)	// 戻り値の型がコンパイルエラー 
\end{lstlisting}

\begin{lstlisting}[language=scala, label=src:higher_order_function, caption=関数をメソッドの引数や戻り値に利用する]
// 関数を引数に取る関数
def debugPrint(x:Int, func:(Int) => Int) {
  val result = func(x)
  println(result)
}
debugPrint(10, square) // 100が表示される
 
// デバッグ版のsquareメソッドを返す関数
def newSquareDebugFunc = {
  val squareDebugFunc = (x:Int) => {
    val result = square(x) println(result)
    result
  }
  squareDebugFunc
}
val squareDebug = newSquareDebugFunc
val result = squareDebug(5) // 25が表示される 
\end{lstlisting}

\section{ScalaとJavaのコレクションの違い}
ここで、関数と深い関係にあるコレクションについて説明します。Javaとの違いが際立つListとArrayについて見ていきましょう。

\subsection{Listの生成と追加、順次処理を学ぶ}
ScalaのListは、JavaのListと同様、順序を持つコレクションです。JavaのListと大きく異なるのは、Listは「不変オブジェクト」であるという点です。例えば、一度作成したListへの要素の追加や削除はできません。これは関数型言語の特徴である、「副作用」がないプログラミングスタイルに基づきます。既存のインスタンスに要素を追加するのではなく、要素が追加された新しいインスタンスを返すことになっています\footnote{List同士の連結には++演算子以外に:::演算子も利用可能です。また、Listから不要な要素を取り除いた新たなListを得るにはfiliterNotメソッドを使うとよいでしょう。}(\lstRef{src:add_elements})。

それでは、このListの各要素を表示する処理を記述してみましょう。二通りの書き方を示します。一つはJavaなどと同様の命令型の考え方に基づく書き方(\lstRef{src:for_statement})、もう一つが拡張for文と同じように要素を順次取り出す書き方です(\lstRef{src:ex_for_statement})。

\begin{lstlisting}[language=scala, label=src:add_elements, caption=要素を追加する例]
val numbers = List(1, 2, 3)
val newNumbers1 = 0 +: numbers // 先頭に0に追加。 List(0, 1, 2, 3)
val newNumbers2 = 0 :: numbers // 先頭に0に追加。 List(0, 1, 2, 3)
val newNumbers3 = numbers ++ List(4, 5)
// ↑ 末尾にList(4, 5)を追加。 List(1, 2, 3, 4, 5)
val newNumbers4 = numbers :+ 4 // 末尾に4を追加。 List(1, 2, 3, 4) 
\end{lstlisting}

\begin{lstlisting}[language=scala, label=src:for_statement, caption=Listの各要素を表示する二通りの方法(Javaのfor文に相当する書き方)]
val numbers = List(1, 2, 3, 4, 5)
// Javaの"for(int i=0;i<numbers.length;i++){"に相当
for(i <- 0 until numbers.length) {
  println(numbers(i)) // 添え字で要素を取得
}
\end{lstlisting}

\begin{lstlisting}[language=scala, label=src:ex_for_statement, caption=Listの各要素を表示する二通りの方法(Javaの拡張for文に相当する書き方)]
val numbers = List(1, 2, 3, 4, 5)
// Javaの"for(element : numbers){"に相当
for(number <- numbers) {
  println(number)
} 
\end{lstlisting}

\subsection{高階関数を使って繰り返し処理を作る}
Listにはforeachメソッドという高階関数があります。\lstRef{src:list_foreach}の例では、foreachメソッドの引数には、無名関数を与えています。この無名関数の型は、その要素型であるIntを引数に取り、戻り値はUnitです。これを要素ごとに呼び出して繰り返し処理を実現します。

\begin{lstlisting}[language=scala, label=src:list_foreach, caption=Listのforeachメソッドを使って要素を表示する]
val numbers = List(1, 2, 3, 4, 5)
numbers.foreach((number:Int) => println(number))
\end{lstlisting}

さらに、この関数の引数のInt型は、型推論によって省略して次のように記述できます。また、括弧(かっこ)も省略できます。

\begin{lstlisting}[language=scala, frame=none]
numbers.foreach((number) => println(number))
numbers.foreach(number => println(number))
\end{lstlisting}

引数がコードブロック中に一度しか登場しない場合は、 プレースホルダであるアンダースコアを使って、以下のように簡潔に記述できます。一度しかプレースホルダが登場しない場合はプレースホルダ自体を省略できます。

\begin{lstlisting}[language=scala, frame=none]
numbers.foreach(println(_))
numbers.foreach(println) // _を省略
\end{lstlisting}

foreachメソッドに渡す関数の処理が複数行ある場合は、メソッドの括弧を中括弧\{ \}に変えて記述できます。
\subsection{配列にはArrayクラスを使う}
Javaでの配列はint[]やObject[]などで記述しますが、ScalaではArrayというクラスになっています。Arrayの生成例は以下の通りです。

\begin{lstlisting}[language=scala, frame=none]
val fruits = Array("apple", "banana")
\end{lstlisting}

要素を取得する場合、Javaでは配列の添え字をカギ括弧[ ]で囲みますが、Scalaではapplyメソッドへの糖衣構文になっている丸括弧( )を利用します。

\begin{lstlisting}[language=scala, frame=none]
val fruit1 = fruits(1) // banana
// ↑fruits.apply(1)の糖衣構文
val fruit2 = fruits(2) // 例外が発生
\end{lstlisting}

Javaの配列と同様、Arrayは要素を更新できます。要素の取得と同様、添え字は丸括弧で指定しますが、これはupdateメソッドへの糖衣構文になっています。

\begin{lstlisting}[language=scala, frame=none]
fruits(1) = "cherry"
// ↑fruits.update(1,"cherry")の糖衣構文 最後に、要素を繰り返し処理する場合を以下に示します。
 
// for版 
for(fruit <- fruits){
  println(fruit)
}
 
// foreach版
fruits.foreach(println)
\end{lstlisting}

\section{Scala プログラミングに欠かせない機能}
ここからは、Scalaプログラミングに欠かせない「パターンマッチ」と、独自機能である「ケースクラス」、そして「トレイト」について説明します。 

\subsection{match式による柔軟なパターンマッチ}
Scalaのパターンマッチは、Javaのswitch文をより強化し た場合分けの機能を提供します。パターンマッチは多くの言 語で採用されていますが、Scalaのものは様々な条件にマッチさせることができ、Scalaプログラミングの中心的な役割を担います。パターンマッチの処理を記述するにはmatch式を利用します。

\subsection{数値や文字列などでマッチング可能}
数値でのマッチの例が\lstRef{src:digit_matching}です。変数n(セレクターと呼びます)に対して、match式を適用し、caseによる選択肢は上から下へと順番に検証されます。Javaのswitch文では breakを記述しなければ後続の選択肢の処理を実行しますが、match式ではbreakはありません。マッチしたcaseを処理したら、他のcaseを処理せずにmatch式の制御を抜けます。 例えば、この例では、case 1で変数nが1に該当すれば\verb|=>|の次の処理のprintln("one") が実行されます。また、case \_はワイルドカードといって、switch文のdefault:に相当します。また、Scalaのmatch式では、文字列やコレクション、型によるマッチングが可能です。\lstRef{src:match_expression}の例ではInt型や文字列の値、List型の条件を指定して文字列を返しています。\lstRef{src:match_expression}で見たように、match式では戻り値を返すことができます。\verb|=>|の後で最後に評価された式の値を戻り値として返せます。

\begin{lstlisting}[language=scala, label=src:digit_matching, caption=数値のマッチングの例]
def numberMatch(n:Int) = n match {
  case 1 => println("one")
  case 2 | 3 => println("two or three")
  case _ => println("other")
}
numberMatch(1) // one
numberMatch(2) // two or three
numberMatch(3) // two or three
numberMatch(4) // other 
\end{lstlisting}

\begin{lstlisting}[language=scala, label=src:match_expression, caption=match式は値を返す]
def convertNumberToString(obj:Any) = obj match {
  case n:Int => "one" // Int型の場合
  case "2" => "two" // 文字列の値の場合
  case List(3,3,3) => "three" // コレクションの場合
  case _ => throw new IllegalArgumentException
}
println(convertNumberToString(1)) // one
println(convertNumberToString("2")) // two
println(convertNumberToString(List(3,3,3)) // three
println(convertNumberToString(0.5)) // 例外発生 
\end{lstlisting}

\begin{lstlisting}[language=scala, label=src:case_class, caption=ケースクラスで定義したバリューオブジェクトの利用]
val p1 = PersonName("Junichi", "Kato")
val p2 = PersonName("JUNICHI", "KATO")
println(p1 == p2) // p1.equals(p2)が呼ばれる

val personName = PersonName("Junichi", "Kato")
val PersonName(f, l) = personName // フィールドを抽出
println("firstName = %s, lastName = %s".format(f, l))
// ↑ firstName = Junichi, lastName = Kato

val PersonName(_, l2) = personName // lastNameだけを抽出
println("lastName = %s".format(l2)) // lastName = Kato 
\end{lstlisting}

\subsection{ケースクラスでバリューオブジェクトを簡単実装}
ケースクラスは、バリューオブジェクト(値を表現するオブジェクト)を実装するために必要な機能を提供するクラスです。例えば、人の名前を表す人名というバリューオブジェクトは次のように記述できます。
\begin{lstlisting}[language=scala, frame=none]
case class PersonName(firstName: String, lastName: String)
\end{lstlisting}
このケースクラスを宣言するとコンパイラが以下を自動的に追加します。

\begin{itemize}
\item ケースクラスに対応するコンパニオンオブジェクトとファクトリメソッドを追加する
\item ケースクラスのコンストラクタの引数宣言をvalで行う
\item toString、hashCode、equalsメソッドの実装を行う
\end{itemize}
 
このため、たった1行の記述で、\lstRef{src:case_class}のようにバリューオブジェクトを扱うことが可能です。 

\subsection{トレイトは実装コードを書けるインタフェース}
ScalaにはJavaのインタフェースと同じ機能はありません。Scalaではインタフェースの代わりにトレイト(trait)を使います。このトレイトを一言でいうと、実装コードを書けるインタフェースのようなものです。 \lstRef{src:trait}は英語と日本でのあいさつを使い分けるアプリをトレイトで実装したサンプルです。GreetingトレイトはJavaのインタフェースのように実装を持ちません。抽象クラスであるAbstractGreetingクラスが継承しています\footnote{二つ以上のトレイトを継承する場合は「... extends Greeting with A with B」 などとwithでつなげて指名します。} 。次はトレイトを使った「ミックスイン」の例です。複数のトレイトを合成することをミックスインと呼びます。\lstRef{src:trait_mixin}を見てください。今度はトレイトに処理が実装されています。トレイトにはコンストラクタが記述できない以外は、クラスと同様に実装コードも記述できます。ここでは、従業員(Employee)をメモリー上で読み書きできるリポジトリを例に挙げます。リポジトリの抽象クラスである EmployeeRepositoryBaseは内部にMapを持ち、ListやMapへの変換メソッドを提供します。そして、リポジトリとして書き込みや読み込みの機能を提供するWritableEmployeeRepositoryトレイトとReadableEmployeeRepositoryトレイトがあります。先ほどの例と違い実装コードを含みます。また、これらのトレイトではthis型でEmployeeRepositoryBaseクラスを指定しており、ミックスイン対象のEmployeeRepositoryBaseの機能にアクセスできます。EmployeeReadOnlyRepositoryクラスは読み込み専用で、EmployeeRepositoryクラスは読み書きができ、toReadableメソッドでEmployeeReadOnlyRepositoryクラスに変換することがで きます。トレイトでミックスインを使うとクラスに実装する処理を小さな断片として定義し、必要なときに組み合わせてクラスを実装することが可能です。JavaにはないScalaの魅力の一つです。 

\begin{lstlisting}[language=scala, label=src:trait, caption=トレイトの利用例。ScalaにはJavaのinterfaceに相当する機能はなく、トレイトを使用する]
// あいさつのトレイト
trait Greeting {
  def getMessage: String
}
// あいさつの抽象クラス
abstract class AbstractGreeting(target:String) extends Greeting {
  def getMessageFormat:String
  override def getMessage = getMessageFormat.format(target)
}
// 日本語のあいさつ
class GreetingJapanese(target:String) extends AbstractGreeting(target) {
  override def getMessageFormat = "%sさん、こんにちは"
}
// 英語のあいさつ
class GreetingEnglish(target:String) extends AbstractGreeting(target) {
  override def getMessageFormat = "%s, Hello"
}
// あいさつするアプリケーション
object GreetingApplication {
  // ポリモーフィズムを使ってあいさつを表示
  private def say(greeting: Greeting) {
    println(greeting.getMessage)
  }
  def main(args: Array[String]) {
    args match {
      case Array(lang, user) =>
          lang match {
            case "JP" => say(new GreetingJapanese(user))
            case "EN" => say(new GreetingEnglish(user))
            case _ => println("指定されたあいさつは実行できません。")
          }
        case _ => println("引数の数が不正です。") }
      }
    }
  }
}
\end{lstlisting}

\begin{lstlisting}[language=scala, label=src:trait_mixin, caption=複数のトレイトを使ったミックスインの例]
package repository
// 従業員
case class Employee(id: Int, name: String)
// 従業員情報を保持するリポジトリの骨格実装
abstract class EmployeeRepositoryBase {
  private[repository] var entities =
    Map.empty[Int, Employee]
  def toMap = entities
  def toList = entities.toList
}
// 読み込み専用のリポジトリ機能
trait ReadableEmployeeRepository {
  this: EmployeeRepositoryBase => // this型の指定
  def load(key: Int): Employee = entities(key)
}
  
// 書き込み専用のリポジトリ機能
trait WritableEmployeeRepository {
  this: EmployeeRepositoryBase => // this型の指定
  def save(employee: Employee): Unit =
    entities += (employee.id -> employee)
}
// 読み込み専用リポジトリの宣言
class EmployeeReadOnlyRepository
  extends EmployeeRepositoryBase
  with ReadableEmployeeRepository
// 読み書き可能なリポジトリの宣言
class EmployeeRepository
  extends EmployeeRepositoryBase
  with WritableEmployeeRepository
  with ReadableEmployeeRepository {

  def toReadable = {
    val result = new EmployeeReadOnlyRepository
    result.entities ++= entities
    result
  }
   
}
// アプリケーション本体
object RepositoryApplication {
  def main(args: Array[String]) {
    val repos = new EmployeeRepository
    repos.save(Employee(1, "KATO"))
    val emp = repos.load(1)
    println(emp) // Employee(1, KATO)
     
    val readOnlyRepos = repos.toReadable
    val emp2 = readOnlyRepos.load(1)
    println(emp2) // Employee(1, KATO))
  }
} 
\end{lstlisting} 

Scalaには次世代言語として様々な機能が実装されています。言語としての機能が多いとどうしても敬遠しがちですが、まずは自分が理解した範囲からScalaを使い始めてもよいと思います。関数型が難しければ、従来 からなじみのある命令型でプログラミングしてもScalaの魅力を十分に体感することがで きます。ここではScalaの魅力のほんの一部しか紹介できませんが、これをきっかけにして、Scala関連書籍やネット上のコンテンツを合わせて自分なりに学習することをお勧めします。
