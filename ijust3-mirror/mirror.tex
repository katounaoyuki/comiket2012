\section{はじめに}

ゲームの体験版やデモムービーをダウンロードする際に時々見かけるミラーサイト。そういえば、いつも似たようなサイト名やサイトバナーが並んでいるような。

皆様も少し気になりませんか?(図\ref{mirrors})

 

そんなちょっとした興味が切っ掛けとなり、2009年から「Magic Mirror」というサイト名でゲーム関連のミラーサイトの運営を行ってみました。\cite{MagicMirror}

その運営を通して知ったこと、感じたことについて簡単に書いてみました。

 

\section{ミラーサイトの運用}

\subsection{ゲーム関連のミラーサイトとは?}

通常、「ミラーサイト」と言う場合、あるWebサイトの内容を全く同内容で複製したWebサイトを意味しますが、

本記事で扱うゲーム関連のミラーサイトは、体験版やデモムービー等の特定の配布データ(素材)だけを複製し、再配信するWebサイトを指します。

ゲームメーカさん等の公式サイトでも概ね「ミラーサイト」と表記され、配布データやダウンロードページへのリンクが貼られています。(他にも「ダウンロード支援サイト」、「協力サイト」等も表記とされることもあります。)

 

\subsection{ミラーサイトの活動と目的}

ゲーム情報サイトや販売店のWebサイトで、コンテンツの一つとしてミラーを行う場合もあるため

例えば独自にゲーム情報をコンテンツとして掲載したり、販売店のWebサイトでダウンロード支援を行う等、各々のミラーサイトの活動や目的は多岐に渡りますが、

単純に「素材の再配信」だけを考える場合、ミラーサイトの活動と目的は次の2つに分類出来ます。\cite{fuzzy2_2}

\begin{enumerate}

 \item ミラー

  負荷分散を目的に各ミラーサイトで再配信を行います。

  古参のミラーサイトであるmirror.fuzzy2.comさんが指摘されるように、現在でも人気の高いファイルは公開直後から数週間程の間、大量のアクセスが掛かります。\cite{fuzzy2_1}

  図\ref{}はその一例として前作がアニメ化された人気タイトルが去年11月に体験版を出した際のダウンロード数を日次で集計したものですが、公開直後の2週間程度で比較的多めのアクセスが続き、その後アクセスが落ち着いて行く様子が見て取れます。(休日やその前日等にアクセスが増えるため、グラフは山なりになっています)

 

 \item 転載

  負荷分散以外の様々な理由で、ミラーサイト側からメーカへ許諾を取り、再配信を行います。

  例えば、メーカがデモムービーを動画サイトのみで公開するような時に、メーカへ確認を取り、ミラーサイトで素材をダウンロード出来るようにする場合があります

\end{enumerate}

 

 

\subsection{KEL有志ミラーサイト}

 

 

 

 

\subsection{ミラー開始までの流れ}
- ゲームメーカや同人サークルからの依頼・または許諾を受けてゲーム販促素材を配布
- ゲーム販促素材はゲームの体験版やデモムービー、主題歌・挿入歌等の音楽が多い。有料でない

\section{ミラーサイトの運用環境}

 

\subsection{サーバ構成}

 

- 3台構成

- IP limitconn

- 総転送量の制限

- 高帯域な環境で分散

 

 

 

%\subsection{東日本大震災の影響}

%

%- バックアップ超大事

%- 関東に半分程度のサイトが集中

%- 計画停電は自宅サーバの運営に支障

 

\section{謝辞}

今回、初めて同人誌の記事を書かせて頂きました。

書き始めた当初は、PostgreSQLと呼ばれるデータベース管理システムの検索実行処理について解説を書こうと壮大な計画を立てていたのですが、社会人として、仕事との両立の難しさも経験しながら、今回は雑誌の主テーマの技術系とは少し異なるライトな記事での執筆となりました。(書き掛けの記事については是非、次の機会に!)

 

記事の方針について右往左往しまして、多大なご迷惑をお掛けしました共同執筆者の皆様と、そして読者の皆様、最後にMagicMirrorのユーザの皆様へ、ここに記して謝意を表します.