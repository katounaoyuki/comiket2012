\chapter{(ゲーム体験版・デモムービーの)ミラーサイトやってみた!}

 

\section{はじめに}

ゲームの体験版やデモムービーをダウンロードする際に時々見かけるミラーサイト。そういえば、いつも似たようなサイト名やサイトバナーが並んでいるような。%

皆様も少し気になりませんか?(図\ref{mirrors})

 

そんなちょっとした興味が切っ掛けとなり、2009年から「Magic Mirror」というサイト名でゲーム関連のミラーサイトの運営を行ってみました。\cite{MagicMirror}%

今回は、そのサイト運営について簡単に書いてみます。

 

\section{ミラーサイトの日々の運用}

%\subsection{ゲーム関連のミラーサイトとは?}

%通常、「ミラーサイト」と言う場合、あるWebサイトの内容を全く同内容で複製したWebサイトを意味しますが、

%今回扱うゲーム関連のミラーサイトは、体験版やデモムービー等の特定の配布データ(素材)だけを複製し、再配信するWebサイトを指します。

%ゲームメーカ等の公式サイトでも、ダウンロードページで「ミラーサイト」の表記が良く使用されています。%

%(他にも「ダウンロード支援サイト」、「協力サイト」等の表記がされることもあります。)

 

\subsection{ゲーム関連のミラーサイトとは?}

ゲーム関連のミラーサイトとは、ゲームメーカや同人のゲームサークルが配布する体験版やデモムービー等の素材をミラーし、配布支援を有志で行っているサイトです。

%最近、仕事が忙しく更新が滞ってしまいがちなのですが、去年までの積極的にミラーを行っていた時期は、一日3000~5000PV程度の、個人運営としてはそこそこアクセスのあるサイトになりました。

 

 

\subsection{ミラーサイトの意義}

個人ミラーサイトの歴史についてはmirror.fuzzy2.comさんが簡潔にまとめられていますが。\cite{fuzzy2_1}

日本国内にブロードバンドの普及が急速に進み、ゲームメーカもインターネット上で体験版やデモムービー等の販促素材を配布するようになりました。

> しかし、ユーザー側の回線性能が急激に増強されたのに対し、メーカーサイトや商用ゲーム情報サイトのサーバーの能力アップが追いつかず、サーバーがダウンする問題がしばしば発生しました。

> そして、この問題に対する「ユーザーによる継続的な支援」の一つが個人ミラーサイトです。初期にはメーカーへの許諾確認が困難であったため勝手にミラーを行う状況が一部にあるなど混乱していたようですが、2002年の中頃に一部の個人ミラーサイト間で連携が始まり、メーカーや流通会社へミラーの許可を得るための仕組みが整っていきました。

 

\subsection{KEL有志ミラー}

 

J-NODEという流通メーカ

JDM

KEL

「KEL有志ミラー」と呼ばれる、有志でゲームソフト関連の素材をミラーするサイトの集まり

 許諾取りやすい

MagicMirrorも「KEL有志ミラー」参加のサイトと連携をしながら活動を行っています。

 

 

 

 

 

% KELに参加するミラーサイトは、ゲームメーカやサークルの許諾の下、無料で素材のミラーを行っています。

 

% TODO: KELの活動についてほげほげ

% 各ミラーサイトではメーカからの許諾を基にミラーを行いますが、これを各ミラーサイトがメーカと全て個別にやり取りすることは現実的に難しく、KEL有志ミラー

 

%- ゲームメーカや同人サークルからの依頼・または許諾を受けてゲーム販促素材を配布

%- ゲーム販促素材はゲームの体験版やデモムービー、主題歌・挿入歌等の音楽が多い。有料でない

 

 

 

\subsection{ミラーサイトの活動と目的}

各々のミラーサイトの活動は、独自色を持つサイトも多く、多岐に渡りますが、%

単純に「素材の再配信」だけを考える場合、ミラーサイトの活動と目的は次の2つに分類出来ます。\cite{fuzzy2_2}

\begin{enumerate}

 \item ミラー

  負荷分散を目的に各ミラーサイトで再配信を行います。

  古参のミラーサイトであるmirror.fuzzy2.comさんが指摘されるように、現在でも人気の高いファイルは公開直後から数週間程の間、大量のアクセスが掛かります。\cite{fuzzy2_1}

  図\ref{}はその一例として前作がアニメ化された人気タイトルが去年11月に体験版を出した際のダウンロード数を日次で集計したものですが、公開直後の2週間程度で比較的多めのアクセスが続き、その後アクセスが落ち着いて行く様子が見て取れます。\footnote{休日やその前日等にアクセスが増えるため、グラフは山なりになっています}

 

 \item 転載

  負荷分散以外の様々な理由で、ミラーサイト側からメーカへ許諾を取り、再配信を行います。

  例えば、メーカがデモムービーを動画共有サービスのみで公開を開始するような時に、メーカへ確認を取り、ミラーサイトでユーザが素材をダウンロード出来るようにする等の場合があります。

\end{enumerate}

 

 

 

 

\subsection{ミラー開始までの流れ}

\begin{enumerate}

 \item メーカやサークルからのミラーの依頼、または許諾申請

 \item KEL有志ミラーメーリングでの募集・情報公開

 \item 各ミラーサイトでのファイルの取得・設置

 \item 幹事がメーカ・サークルへミラーを行ったサイトの情報を報告

 \item ミラー公開

\end{enumerate}

\cite{XES}

 

% --------------------------------------------------------

 

\section{ミラーサイトの運用環境}

%ゲーム関連のミラーサイトの運用環境の基本は、単純に配布対象のファイルをWeb上で公開しているだけですので、

%KEL有志ミラーでは、特に

本節では、MagicMirrorで使用している運用環境について紹介します。

 

\subsection{サーバ構成}

 

\begin{itemize}

 \item [VPS] Webアプリケーションサーバ + DBサーバ (ミラー素材の索引・検索・ダウンロードページなどを提供)

 \item [VPS] ファイルサーバ

 \item [自宅] ファイルサーバ

\end{itemize}

 

\subsection{アクセス制限と時限バッチ}

 

 

\subsection{転送の制限について}

- IP limitconn

- 総転送量の制限

 

 

\subsection{負荷分散について}

- 高帯域な環境で分散

 

 

\subsection{集計処理}

ダウンロード数

 

%\subsection{東日本大震災の影響}

%

%- バックアップ超大事

%- 関東に半分程度のサイトが集中

%- 計画停電は自宅サーバの運営に支障

 

\section{謝辞}

今回、初めて同人誌の記事を書かせて頂きました。

書き始めた当初は、PostgreSQLと呼ばれるデータベース管理システムの検索実行処理について解説を書こうと壮大な計画を立てていたのですが、社会人として、仕事との両立の難しさも経験しながら、今回は雑誌の主テーマの技術系とは少し異なるライトな記事での執筆となりました。(書き掛けの記事については是非、次の機会に!)

 

記事の方針について右往左往しまして、多大なご迷惑をお掛けしました共同執筆者の皆様と、そして読者の皆様、最後にMagicMirrorのユーザの皆様へ、ここに記して謝意を表します.